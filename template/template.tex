% Options for packages loaded elsewhere
\PassOptionsToPackage{unicode}{hyperref}
\PassOptionsToPackage{hyphens}{url}
\PassOptionsToPackage{dvipsnames,svgnames,x11names}{xcolor}
%
\documentclass[
  letterpaper,
]{article}

\usepackage{amsmath,amssymb}
\usepackage{iftex}
\ifPDFTeX
  \usepackage[T1]{fontenc}
  \usepackage[utf8]{inputenc}
  \usepackage{textcomp} % provide euro and other symbols
\else % if luatex or xetex
  \usepackage{unicode-math}
  \defaultfontfeatures{Scale=MatchLowercase}
  \defaultfontfeatures[\rmfamily]{Ligatures=TeX,Scale=1}
\fi
\usepackage{lmodern}
\ifPDFTeX\else  
    % xetex/luatex font selection
\fi
% Use upquote if available, for straight quotes in verbatim environments
\IfFileExists{upquote.sty}{\usepackage{upquote}}{}
\IfFileExists{microtype.sty}{% use microtype if available
  \usepackage[]{microtype}
  \UseMicrotypeSet[protrusion]{basicmath} % disable protrusion for tt fonts
}{}
\makeatletter
\@ifundefined{KOMAClassName}{% if non-KOMA class
  \IfFileExists{parskip.sty}{%
    \usepackage{parskip}
  }{% else
    \setlength{\parindent}{0pt}
    \setlength{\parskip}{6pt plus 2pt minus 1pt}}
}{% if KOMA class
  \KOMAoptions{parskip=half}}
\makeatother
\usepackage{xcolor}
\setlength{\emergencystretch}{3em} % prevent overfull lines
\setcounter{secnumdepth}{5}
% Make \paragraph and \subparagraph free-standing
\makeatletter
\ifx\paragraph\undefined\else
  \let\oldparagraph\paragraph
  \renewcommand{\paragraph}{
    \@ifstar
      \xxxParagraphStar
      \xxxParagraphNoStar
  }
  \newcommand{\xxxParagraphStar}[1]{\oldparagraph*{#1}\mbox{}}
  \newcommand{\xxxParagraphNoStar}[1]{\oldparagraph{#1}\mbox{}}
\fi
\ifx\subparagraph\undefined\else
  \let\oldsubparagraph\subparagraph
  \renewcommand{\subparagraph}{
    \@ifstar
      \xxxSubParagraphStar
      \xxxSubParagraphNoStar
  }
  \newcommand{\xxxSubParagraphStar}[1]{\oldsubparagraph*{#1}\mbox{}}
  \newcommand{\xxxSubParagraphNoStar}[1]{\oldsubparagraph{#1}\mbox{}}
\fi
\makeatother

\usepackage{color}
\usepackage{fancyvrb}
\newcommand{\VerbBar}{|}
\newcommand{\VERB}{\Verb[commandchars=\\\{\}]}
\DefineVerbatimEnvironment{Highlighting}{Verbatim}{commandchars=\\\{\}}
% Add ',fontsize=\small' for more characters per line
\usepackage{framed}
\definecolor{shadecolor}{RGB}{241,243,245}
\newenvironment{Shaded}{\begin{snugshade}}{\end{snugshade}}
\newcommand{\AlertTok}[1]{\textcolor[rgb]{0.68,0.00,0.00}{#1}}
\newcommand{\AnnotationTok}[1]{\textcolor[rgb]{0.37,0.37,0.37}{#1}}
\newcommand{\AttributeTok}[1]{\textcolor[rgb]{0.40,0.45,0.13}{#1}}
\newcommand{\BaseNTok}[1]{\textcolor[rgb]{0.68,0.00,0.00}{#1}}
\newcommand{\BuiltInTok}[1]{\textcolor[rgb]{0.00,0.23,0.31}{#1}}
\newcommand{\CharTok}[1]{\textcolor[rgb]{0.13,0.47,0.30}{#1}}
\newcommand{\CommentTok}[1]{\textcolor[rgb]{0.37,0.37,0.37}{#1}}
\newcommand{\CommentVarTok}[1]{\textcolor[rgb]{0.37,0.37,0.37}{\textit{#1}}}
\newcommand{\ConstantTok}[1]{\textcolor[rgb]{0.56,0.35,0.01}{#1}}
\newcommand{\ControlFlowTok}[1]{\textcolor[rgb]{0.00,0.23,0.31}{\textbf{#1}}}
\newcommand{\DataTypeTok}[1]{\textcolor[rgb]{0.68,0.00,0.00}{#1}}
\newcommand{\DecValTok}[1]{\textcolor[rgb]{0.68,0.00,0.00}{#1}}
\newcommand{\DocumentationTok}[1]{\textcolor[rgb]{0.37,0.37,0.37}{\textit{#1}}}
\newcommand{\ErrorTok}[1]{\textcolor[rgb]{0.68,0.00,0.00}{#1}}
\newcommand{\ExtensionTok}[1]{\textcolor[rgb]{0.00,0.23,0.31}{#1}}
\newcommand{\FloatTok}[1]{\textcolor[rgb]{0.68,0.00,0.00}{#1}}
\newcommand{\FunctionTok}[1]{\textcolor[rgb]{0.28,0.35,0.67}{#1}}
\newcommand{\ImportTok}[1]{\textcolor[rgb]{0.00,0.46,0.62}{#1}}
\newcommand{\InformationTok}[1]{\textcolor[rgb]{0.37,0.37,0.37}{#1}}
\newcommand{\KeywordTok}[1]{\textcolor[rgb]{0.00,0.23,0.31}{\textbf{#1}}}
\newcommand{\NormalTok}[1]{\textcolor[rgb]{0.00,0.23,0.31}{#1}}
\newcommand{\OperatorTok}[1]{\textcolor[rgb]{0.37,0.37,0.37}{#1}}
\newcommand{\OtherTok}[1]{\textcolor[rgb]{0.00,0.23,0.31}{#1}}
\newcommand{\PreprocessorTok}[1]{\textcolor[rgb]{0.68,0.00,0.00}{#1}}
\newcommand{\RegionMarkerTok}[1]{\textcolor[rgb]{0.00,0.23,0.31}{#1}}
\newcommand{\SpecialCharTok}[1]{\textcolor[rgb]{0.37,0.37,0.37}{#1}}
\newcommand{\SpecialStringTok}[1]{\textcolor[rgb]{0.13,0.47,0.30}{#1}}
\newcommand{\StringTok}[1]{\textcolor[rgb]{0.13,0.47,0.30}{#1}}
\newcommand{\VariableTok}[1]{\textcolor[rgb]{0.07,0.07,0.07}{#1}}
\newcommand{\VerbatimStringTok}[1]{\textcolor[rgb]{0.13,0.47,0.30}{#1}}
\newcommand{\WarningTok}[1]{\textcolor[rgb]{0.37,0.37,0.37}{\textit{#1}}}

\providecommand{\tightlist}{%
  \setlength{\itemsep}{0pt}\setlength{\parskip}{0pt}}\usepackage{longtable,booktabs,array}
\usepackage{calc} % for calculating minipage widths
% Correct order of tables after \paragraph or \subparagraph
\usepackage{etoolbox}
\makeatletter
\patchcmd\longtable{\par}{\if@noskipsec\mbox{}\fi\par}{}{}
\makeatother
% Allow footnotes in longtable head/foot
\IfFileExists{footnotehyper.sty}{\usepackage{footnotehyper}}{\usepackage{footnote}}
\makesavenoteenv{longtable}
\usepackage{graphicx}
\makeatletter
\newsavebox\pandoc@box
\newcommand*\pandocbounded[1]{% scales image to fit in text height/width
  \sbox\pandoc@box{#1}%
  \Gscale@div\@tempa{\textheight}{\dimexpr\ht\pandoc@box+\dp\pandoc@box\relax}%
  \Gscale@div\@tempb{\linewidth}{\wd\pandoc@box}%
  \ifdim\@tempb\p@<\@tempa\p@\let\@tempa\@tempb\fi% select the smaller of both
  \ifdim\@tempa\p@<\p@\scalebox{\@tempa}{\usebox\pandoc@box}%
  \else\usebox{\pandoc@box}%
  \fi%
}
% Set default figure placement to htbp
\def\fps@figure{htbp}
\makeatother
% definitions for citeproc citations
\NewDocumentCommand\citeproctext{}{}
\NewDocumentCommand\citeproc{mm}{%
  \begingroup\def\citeproctext{#2}\cite{#1}\endgroup}
\makeatletter
 % allow citations to break across lines
 \let\@cite@ofmt\@firstofone
 % avoid brackets around text for \cite:
 \def\@biblabel#1{}
 \def\@cite#1#2{{#1\if@tempswa , #2\fi}}
\makeatother
\newlength{\cslhangindent}
\setlength{\cslhangindent}{1.5em}
\newlength{\csllabelwidth}
\setlength{\csllabelwidth}{3em}
\newenvironment{CSLReferences}[2] % #1 hanging-indent, #2 entry-spacing
 {\begin{list}{}{%
  \setlength{\itemindent}{0pt}
  \setlength{\leftmargin}{0pt}
  \setlength{\parsep}{0pt}
  % turn on hanging indent if param 1 is 1
  \ifodd #1
   \setlength{\leftmargin}{\cslhangindent}
   \setlength{\itemindent}{-1\cslhangindent}
  \fi
  % set entry spacing
  \setlength{\itemsep}{#2\baselineskip}}}
 {\end{list}}
\usepackage{calc}
\newcommand{\CSLBlock}[1]{\hfill\break\parbox[t]{\linewidth}{\strut\ignorespaces#1\strut}}
\newcommand{\CSLLeftMargin}[1]{\parbox[t]{\csllabelwidth}{\strut#1\strut}}
\newcommand{\CSLRightInline}[1]{\parbox[t]{\linewidth - \csllabelwidth}{\strut#1\strut}}
\newcommand{\CSLIndent}[1]{\hspace{\cslhangindent}#1}

\usepackage{orcidlink}  
\usepackage[nonatbib, preprint]{neurips_2023}
\usepackage{amsmath}
\usepackage{amsthm}
\usepackage[utf8]{inputenc} % allow utf-8 input
\usepackage[T1]{fontenc}    % use 8-bit T1 fonts
\usepackage{hyperref}       % hyperlinks
\usepackage{url}            % simple URL typesetting
\usepackage{booktabs}       % professional-quality tables
\usepackage{amsfonts}       % blackboard math symbols
\usepackage{nicefrac}       % compact symbols for 1/2, etc.
\usepackage{microtype}      % microtypography
\usepackage{xcolor}         % colors
\usepackage[numbers]{natbib}
\usepackage{comment}
\usepackage{graphicx} 
\bibliographystyle{abbrvnat}
\makeatletter
\@ifpackageloaded{tcolorbox}{}{\usepackage[skins,breakable]{tcolorbox}}
\@ifpackageloaded{fontawesome5}{}{\usepackage{fontawesome5}}
\definecolor{quarto-callout-color}{HTML}{909090}
\definecolor{quarto-callout-note-color}{HTML}{0758E5}
\definecolor{quarto-callout-important-color}{HTML}{CC1914}
\definecolor{quarto-callout-warning-color}{HTML}{EB9113}
\definecolor{quarto-callout-tip-color}{HTML}{00A047}
\definecolor{quarto-callout-caution-color}{HTML}{FC5300}
\definecolor{quarto-callout-color-frame}{HTML}{acacac}
\definecolor{quarto-callout-note-color-frame}{HTML}{4582ec}
\definecolor{quarto-callout-important-color-frame}{HTML}{d9534f}
\definecolor{quarto-callout-warning-color-frame}{HTML}{f0ad4e}
\definecolor{quarto-callout-tip-color-frame}{HTML}{02b875}
\definecolor{quarto-callout-caution-color-frame}{HTML}{fd7e14}
\makeatother
\makeatletter
\@ifpackageloaded{caption}{}{\usepackage{caption}}
\AtBeginDocument{%
\ifdefined\contentsname
  \renewcommand*\contentsname{Table of contents}
\else
  \newcommand\contentsname{Table of contents}
\fi
\ifdefined\listfigurename
  \renewcommand*\listfigurename{List of Figures}
\else
  \newcommand\listfigurename{List of Figures}
\fi
\ifdefined\listtablename
  \renewcommand*\listtablename{List of Tables}
\else
  \newcommand\listtablename{List of Tables}
\fi
\ifdefined\figurename
  \renewcommand*\figurename{Figure}
\else
  \newcommand\figurename{Figure}
\fi
\ifdefined\tablename
  \renewcommand*\tablename{Table}
\else
  \newcommand\tablename{Table}
\fi
}
\@ifpackageloaded{float}{}{\usepackage{float}}
\floatstyle{ruled}
\@ifundefined{c@chapter}{\newfloat{codelisting}{h}{lop}}{\newfloat{codelisting}{h}{lop}[chapter]}
\floatname{codelisting}{Listing}
\newcommand*\listoflistings{\listof{codelisting}{List of Listings}}
\usepackage{amsthm}
\theoremstyle{plain}
\newtheorem{theorem}{Theorem}[section]
\theoremstyle{remark}
\AtBeginDocument{\renewcommand*{\proofname}{Proof}}
\newtheorem*{remark}{Remark}
\newtheorem*{solution}{Solution}
\newtheorem{refremark}{Remark}[section]
\newtheorem{refsolution}{Solution}[section]
\makeatother
\makeatletter
\makeatother
\makeatletter
\@ifpackageloaded{caption}{}{\usepackage{caption}}
\@ifpackageloaded{subcaption}{}{\usepackage{subcaption}}
\makeatother

\usepackage{bookmark}

\IfFileExists{xurl.sty}{\usepackage{xurl}}{} % add URL line breaks if available
\urlstyle{same} % disable monospaced font for URLs
\hypersetup{
  pdftitle={STATS 531 project report format},
  pdfauthor={Blinded},
  colorlinks=true,
  linkcolor={blue},
  filecolor={Maroon},
  citecolor={Blue},
  urlcolor={Blue},
  pdfcreator={LaTeX via pandoc}}



\title{STATS 531 project report format}
\author{
\textbf{Blinded}%
\\%
%
}
\date{2026-01-03}
\begin{document}
\maketitle
\begin{abstract}
This is the abstract. It should briefly describe the context of the
report and highlight the findings of highest interest to the target
readers (other students in the class, the GSI, and the instructor).
\end{abstract}


\section{About this document}\label{about-this-document}

This document provides a template based on the
\href{https://quarto.org/}{quarto system} based on
https://github.com/stevengogogo/neurips-quarto-extension/ and adapted
for STATS 531. We show how \texttt{Python} (Perez, Granger, and Hunter
2011) or \texttt{R} (R Core Team 2020) code can be included.

\section{Formatting}\label{formatting}

This section covers basic formatting guidelines.
\href{https://quarto.org/}{Quarto} is a versatile formatting system for
authoring HTML based on markdown, integrating {\LaTeX} and various code
block interpreted either via Jupyter or Knitr (and thus deal with
Python, R and many other langages). It relies on the
\href{https://rmarkdown.rstudio.com/authoring_pandoc_markdown.html}{Pandoc
Markdown} markup language.

\begin{tcolorbox}[enhanced jigsaw, opacitybacktitle=0.6, arc=.35mm, left=2mm, title=\textcolor{quarto-callout-note-color}{\faInfo}\hspace{0.5em}{Block title 1}, opacityback=0, rightrule=.15mm, breakable, colframe=quarto-callout-note-color-frame, colback=white, bottomrule=.15mm, bottomtitle=1mm, colbacktitle=quarto-callout-note-color!10!white, toprule=.15mm, toptitle=1mm, leftrule=.75mm, coltitle=black, titlerule=0mm]

We will only give some formatting elements. Authors can refer to the
\href{https://quarto.org/}{Quarto web page} for a complete view of the
formatting possibilities.

\end{tcolorbox}

\subsection{Block title 2}\label{block-title-2}

To render/compile a document, run \texttt{quarto\ render}. A document
will be generated that includes both content as well as the output of
any embedded code chunks within the document:

\begin{Shaded}
\begin{Highlighting}[]
\ExtensionTok{quarto}\NormalTok{ render content.qmd }\CommentTok{\# will render to pdf}
\end{Highlighting}
\end{Shaded}

\subsection{Basic markdown formatting}\label{basic-markdown-formatting}

\textbf{Bold text} or \emph{italic}

\begin{itemize}
\tightlist
\item
  This is a list
\item
  With more elements
\item
  It isn't numbered.
\end{itemize}

But we can also do a numbered list

\begin{enumerate}
\def\labelenumi{\arabic{enumi}.}
\tightlist
\item
  This is my first item
\item
  This is my second item
\item
  This is my third item
\end{enumerate}

\subsection{Mathematics}\label{mathematics}

\subsubsection{Mathematical formulae}\label{mathematical-formulae}

\href{https://www.latex-project.org/}{{\LaTeX}} code is natively
supported\footnote{We use \href{https://katex.org/}{katex} for this
  purpose.}, which makes it possible to use mathematical formulae:

will render

\[
f(x_1, \dots, x_n; \mu, \sigma^2) =
\frac{1}{\sigma \sqrt{2\pi}} \exp{\left(- \frac{1}{2\sigma^2}\sum_{i=1}^n(x_i - \mu)^2\right)}
\]

It is also posible to cross-reference an equation, see
Equation~\ref{eq-mylabel}:

\begin{equation}\phantomsection\label{eq-mylabel}{
\begin{aligned}
D_{x_N} & = \frac12
\left[\begin{array}{cc}
x_L^\top & x_N^\top \end{array}\right] \,
\left[\begin{array}{cc}  L_L & B \\ B^\top & L_N \end{array}\right] \,
\left[\begin{array}{c}
x_L \\ x_N \end{array}\right] \\
& = \frac12 (x_L^\top L_L x_L + 2 x_N^\top B^\top x_L + x_N^\top L_N x_N),
\end{aligned}
}\end{equation}

\subsubsection{Theorems and other amsthem-like
environments}\label{theorems-and-other-amsthem-like-environments}

Quarto includes a nice support for theorems, with predefined prefix
labels for theorems, lemmas, proposition, etc. see
\href{https://quarto.org/docs/authoring/cross-references.html\#theorems-and-proofs}{this
page}. Here is a simple example:

\begin{theorem}[Strong law of large
numbers]\protect\hypertarget{thm-slln}{}\label{thm-slln}

The sample average converges almost surely to the expected value:

\[\overline{X}_n\ \xrightarrow{\text{a.s.}}\ \mu \qquad\textrm{when}\ n \to \infty.\]

\end{theorem}

See Theorem~\ref{thm-slln}.

\subsection{Code}\label{code}

Quarto uses either Jupyter or knitr to render code chunks. This can be
triggered in the yaml header, e.g., for Jupyter (should be installed on
your computer) use

\begin{Shaded}
\begin{Highlighting}[]
\PreprocessorTok{{-}{-}{-}}
\FunctionTok{title}\KeywordTok{:}\AttributeTok{ }\StringTok{"My Document"}
\AttributeTok{author "Jane Doe"}
\FunctionTok{jupyter}\KeywordTok{:}\AttributeTok{ python3}
\PreprocessorTok{{-}{-}{-}}
\end{Highlighting}
\end{Shaded}

For knitr (R + knitr must be installed on your computer)

\begin{Shaded}
\begin{Highlighting}[]
\PreprocessorTok{{-}{-}{-}}
\FunctionTok{title}\KeywordTok{:}\AttributeTok{ }\StringTok{"My Document"}
\AttributeTok{author "Jane Doe"}
\PreprocessorTok{{-}{-}{-}}
\end{Highlighting}
\end{Shaded}

You can use Jupyter for Python code and more. And R + KnitR for if you
want to mix R with Python (via the package reticulate Ushey, Allaire,
and Tang (2020)).

\subsubsection{R}\label{r}

\texttt{R} code (R Core Team 2020) chunks may be embedded as follows:

\begin{Shaded}
\begin{Highlighting}[]
\NormalTok{x }\OtherTok{\textless{}{-}} \FunctionTok{rnorm}\NormalTok{(}\DecValTok{10}\NormalTok{)}
\end{Highlighting}
\end{Shaded}

\subsubsection{Python}\label{python}

\subsection{Figures}\label{figures}

Plots can be generated as follows and referenced. See plot
Figure~\ref{fig-logo}:

BROKEN. NEED A SAMPLE PLOT HERE

It is also possible to create figures from static images:

\begin{figure}

\centering{

\pandocbounded{\includegraphics[keepaspectratio]{figures/sfds.png}}

}

\caption{\label{fig-logo}SFdS logo (c.a. 2021)}

\end{figure}%

\subsection{Tables}\label{tables}

\subsubsection{Markdown syntax}\label{markdown-syntax}

Tables (with label: \texttt{@tbl-mylabel} renders
Table~\ref{tbl-mylabel}) can be generated with markdown as follows

\begin{Shaded}
\begin{Highlighting}[]
\NormalTok{| Tables   |      Are      |  Cool |}
\NormalTok{|{-}{-}{-}{-}{-}{-}{-}{-}{-}{-}|:{-}{-}{-}{-}{-}{-}{-}{-}{-}{-}{-}{-}{-}:|{-}{-}{-}{-}{-}{-}:|}
\NormalTok{| col 1 is |  left{-}aligned | $1600 |}
\NormalTok{| col 2 is |    centered   |   $12 |}
\NormalTok{| col 3 is | right{-}aligned |    $1 |}
\NormalTok{: my table caption \{\#tbl{-}mylabel\}}
\end{Highlighting}
\end{Shaded}

\begin{longtable}[]{@{}lcr@{}}
\caption{my table caption}\label{tbl-mylabel}\tabularnewline
\toprule\noalign{}
Tables & Are & Cool \\
\midrule\noalign{}
\endfirsthead
\toprule\noalign{}
Tables & Are & Cool \\
\midrule\noalign{}
\endhead
\bottomrule\noalign{}
\endlastfoot
col 1 is & left-aligned & \$1600 \\
col 2 is & centered & \$12 \\
col 3 is & right-aligned & \$1 \\
\end{longtable}

\subsubsection{List-table filter}\label{list-table-filter}

We also integrate the
\href{https://github.com/pandoc/lua-filters/tree/master/list-table}{list
tables} filter from Pandoc, so that you may alternatively use this
format , easier to write and maintain:

\begin{Shaded}
\begin{Highlighting}[]
\NormalTok{:::list{-}table}
\SpecialStringTok{   * }\NormalTok{{-} row 1, column 1}
\SpecialStringTok{     {-} }\NormalTok{row 1, column 2}
\SpecialStringTok{     {-} }\NormalTok{row 1, column 3}

\SpecialStringTok{   * }\NormalTok{{-} row 2, column 1}
\NormalTok{     {-}}
\SpecialStringTok{     {-} }\NormalTok{row 2, column 3}

\SpecialStringTok{   * }\NormalTok{{-} row 3, column 1}
\SpecialStringTok{     {-} }\NormalTok{row 3, column 2}
\NormalTok{:::}
\end{Highlighting}
\end{Shaded}

\begin{longtable}[]{@{}lll@{}}
\toprule\noalign{}
row 1, column 1 & row 1, column 2 & row 1, column 3 \\
\midrule\noalign{}
\endhead
\bottomrule\noalign{}
\endlastfoot
row 2, column 1 & & row 2, column 3 \\
row 3, column 1 & row 3, column 2 & \\
\end{longtable}

\subsubsection{Table generated from
code}\label{table-generated-from-code}

Table can also be generated by some code, for instance with
\texttt{knitr} here:

\begin{Shaded}
\begin{Highlighting}[]
\NormalTok{knitr}\SpecialCharTok{::}\FunctionTok{kable}\NormalTok{(}\FunctionTok{summary}\NormalTok{(cars), }\AttributeTok{caption =} \StringTok{"Table caption."}\NormalTok{)}
\end{Highlighting}
\end{Shaded}

\begin{longtable}[]{@{}lll@{}}
\caption{Table caption.}\tabularnewline
\toprule\noalign{}
& speed & dist \\
\midrule\noalign{}
\endfirsthead
\toprule\noalign{}
& speed & dist \\
\midrule\noalign{}
\endhead
\bottomrule\noalign{}
\endlastfoot
& Min. : 4.0 & Min. : 2.00 \\
& 1st Qu.:12.0 & 1st Qu.: 26.00 \\
& Median :15.0 & Median : 36.00 \\
& Mean :15.4 & Mean : 42.98 \\
& 3rd Qu.:19.0 & 3rd Qu.: 56.00 \\
& Max. :25.0 & Max. :120.00 \\
\end{longtable}

\subsection{Algorithms}\label{algorithms}

A solution to typeset pseudocode just like you would do with {\LaTeX},
yet with HTML output is to rely on the JavaScript
\href{https://github.com/SaswatPadhi/pseudocode.js}{pseudocode.js}. Your
pseudocode is written inside a
\href{https://quarto.org/docs/authoring/markdown-basics.html\#source-code}{Code
Block} with the \texttt{pseudocode} class. Do not forget the class tag,
that will trigger the rendering process of your pseudo-code. The result
is as follows\footnote{For proper pdf rendering, use
  \href{https://en.wikipedia.org/wiki/Camel_case}{Camel cased} names for
  all \texttt{algorithmic} keywords, not upper case ones, like the
  examples in \texttt{pseudocode.js}'s documentation, which are not
  compatible with LaTeX.}:

\begin{Shaded}
\begin{Highlighting}[]
\InformationTok{\textasciigrave{}\textasciigrave{}\textasciigrave{}pseudocode}
\InformationTok{\#| label: alg{-}quicksort}
\InformationTok{\#| html{-}indent{-}size: "1.2em"}
\InformationTok{\#| html{-}comment{-}delimiter: "//"}
\InformationTok{\#| html{-}line{-}number: true}
\InformationTok{\#| html{-}line{-}number{-}punc: ":"}
\InformationTok{\#| html{-}no{-}end: false}
\InformationTok{\#| pdf{-}placement: "htb!"}
\InformationTok{\#| pdf{-}line{-}number: true}

\InformationTok{\textbackslash{}begin\{algorithm\}}
\InformationTok{\textbackslash{}caption\{Quicksort\}}
\InformationTok{\textbackslash{}begin\{algorithmic\}}
\InformationTok{\textbackslash{}Procedure\{Quicksort\}\{$A, p, r$\}}
\InformationTok{  \textbackslash{}If\{$p \textless{} r$\}}
\InformationTok{    \textbackslash{}State $q = $ \textbackslash{}Call\{Partition\}\{$A, p, r$\}}
\InformationTok{    \textbackslash{}State \textbackslash{}Call\{Quicksort\}\{$A, p, q {-} 1$\}}
\InformationTok{    \textbackslash{}State \textbackslash{}Call\{Quicksort\}\{$A, q + 1, r$\}}
\InformationTok{  \textbackslash{}EndIf}
\InformationTok{\textbackslash{}EndProcedure}
\InformationTok{\textbackslash{}Procedure\{Partition\}\{$A, p, r$\}}
\InformationTok{  \textbackslash{}State $x = A[r]$}
\InformationTok{  \textbackslash{}State $i = p {-} 1$}
\InformationTok{  \textbackslash{}For\{$j = p, \textbackslash{}dots, r {-} 1$\}}
\InformationTok{    \textbackslash{}If\{$A[j] \textless{} x$\}}
\InformationTok{      \textbackslash{}State $i = i + 1$}
\InformationTok{      \textbackslash{}State exchange}
\InformationTok{      $A[i]$ with     $A[j]$}
\InformationTok{    \textbackslash{}EndIf}
\InformationTok{    \textbackslash{}State exchange $A[i]$ with $A[r]$}
\InformationTok{  \textbackslash{}EndFor}
\InformationTok{\textbackslash{}EndProcedure}
\InformationTok{\textbackslash{}end\{algorithmic\}}
\InformationTok{\textbackslash{}end\{algorithm\}}
\InformationTok{\textasciigrave{}\textasciigrave{}\textasciigrave{}}
\end{Highlighting}
\end{Shaded}

\begin{Shaded}
\begin{Highlighting}[]
\NormalTok{\#| label: alg{-}quicksort}
\NormalTok{\#| html{-}indent{-}size: "1.2em"}
\NormalTok{\#| html{-}comment{-}delimiter: "//"}
\NormalTok{\#| html{-}line{-}number: true}
\NormalTok{\#| html{-}line{-}number{-}punc: ":"}
\NormalTok{\#| html{-}no{-}end: false}
\NormalTok{\#| pdf{-}placement: "htb!"}
\NormalTok{\#| pdf{-}line{-}number: true}

\NormalTok{\textbackslash{}begin\{algorithm\}}
\NormalTok{\textbackslash{}caption\{Quicksort\}}
\NormalTok{\textbackslash{}begin\{algorithmic\}}
\NormalTok{\textbackslash{}Procedure\{Quicksort\}\{$A, p, r$\}}
\NormalTok{  \textbackslash{}If\{$p \textless{} r$\}}
\NormalTok{    \textbackslash{}State $q = $ \textbackslash{}Call\{Partition\}\{$A, p, r$\}}
\NormalTok{    \textbackslash{}State \textbackslash{}Call\{Quicksort\}\{$A, p, q {-} 1$\}}
\NormalTok{    \textbackslash{}State \textbackslash{}Call\{Quicksort\}\{$A, q + 1, r$\}}
\NormalTok{  \textbackslash{}EndIf}
\NormalTok{\textbackslash{}EndProcedure}
\NormalTok{\textbackslash{}Procedure\{Partition\}\{$A, p, r$\}}
\NormalTok{  \textbackslash{}State $x = A[r]$}
\NormalTok{  \textbackslash{}State $i = p {-} 1$}
\NormalTok{  \textbackslash{}For\{$j = p, \textbackslash{}dots, r {-} 1$\}}
\NormalTok{    \textbackslash{}If\{$A[j] \textless{} x$\}}
\NormalTok{      \textbackslash{}State $i = i + 1$}
\NormalTok{      \textbackslash{}State exchange}
\NormalTok{      $A[i]$ with     $A[j]$}
\NormalTok{    \textbackslash{}EndIf}
\NormalTok{    \textbackslash{}State exchange $A[i]$ with $A[r]$}
\NormalTok{  \textbackslash{}EndFor}
\NormalTok{\textbackslash{}EndProcedure}
\NormalTok{\textbackslash{}end\{algorithmic\}}
\NormalTok{\textbackslash{}end\{algorithm\}}
\end{Highlighting}
\end{Shaded}

 Algorithm~\ref{alg-quicksort}  is extracted from Chapter 7,
Introduction to Algorithms (3rd edition).

\subsection{Diagrams}\label{diagrams}

In addition of
\href{https://quarto.org/docs/authoring/diagrams.html}{quarto supported
diagrams}, we also support
\href{https://www.overleaf.com/learn/latex/TikZ_package}{tikz} diagrams.
The following example\footnote{This is the new syntax for
  cross-references since quarto 1.4, see
  \href{https://quarto.org/docs/prerelease/1.4/crossref.html}{Crossreferenceable
  elements}} is rendered as follows.

For learning TiKZ, I recommend this website:
\href{https://tikz.dev/}{Tikz examples}.

\begin{figure}

\centering{

\pandocbounded{\includegraphics[keepaspectratio]{template_files/figure-pdf/fig-tikz-1.pdf}}

}

\caption{\label{fig-tikz}}

\end{figure}%

A simple example of a commutative diagram with \(\texttt{tikz}\).

You may refer to it as Figure~\ref{fig-tikz}.

\subsection{Handling references}\label{sec-references}

\subsubsection{Bibliographic references}\label{bibliographic-references}

References are displayed as footnotes using
\href{http://www.bibtex.org/}{BibTeX}, e.g.~\texttt{{[}@computo{]}} will
be displayed as (Computo Team 2021), where \texttt{computo} is the
bibtex key for this specific entry. The bibliographic information is
automatically retrieved from the \texttt{.bib} file specified in the
header of this document (here:\texttt{references.bib}).

\subsubsection{Other cross-references}\label{other-cross-references}

As already (partially) seen, Quarto includes a mecanism similar to the
bibliographic references for sections, equations, theorems, figures,
lists, etc. Have a look at
\href{https://quarto.org/docs/authoring/cross-references.html}{this
page}.

\subsection*{Bibliography}\label{bibliography}
\addcontentsline{toc}{subsection}{Bibliography}

\phantomsection\label{refs}
\begin{CSLReferences}{1}{0}
\bibitem[\citeproctext]{ref-computo}
Computo Team. 2021. {``Computo: Reproducible Computational/Algorithmic
Contributions in Statistics and Machine Learning.''} \emph{Computo}.

\bibitem[\citeproctext]{ref-perez2011python}
Perez, Fernando, Brian E Granger, and John D Hunter. 2011. {``Python: An
Ecosystem for Scientific Computing.''} \emph{Computing in Science\\
\& Engineering} 13 (2): 13--21.

\bibitem[\citeproctext]{ref-R-base}
R Core Team. 2020. \emph{R: A Language and Environment for Statistical
Computing}. Vienna, Austria: R Foundation for Statistical Computing.
\url{https://www.R-project.org/}.

\bibitem[\citeproctext]{ref-R-reticulate}
Ushey, Kevin, JJ Allaire, and Yuan Tang. 2020. \emph{Reticulate:
Interface to Python}. \url{https://github.com/rstudio/reticulate}.

\end{CSLReferences}




\end{document}
